\documentclass[11pt]{article}

\addtolength{\textwidth}{2cm}
\addtolength{\hoffset}{-1cm}

\makeatletter
\renewcommand{\subsection}[1]{\@startsection{subsection}{2}{0pt}%
{\medskipamount}{-4.5 pt}{\textbf}{#1}\hskip-\parindent\textbf{. }}
\makeatother

\usepackage{url}

\newcommand{\coxeter}{{\tt Coxeter}}
\renewcommand{\a}{\alpha}
\renewcommand{\b}{\beta}
\newcommand{\ds}{\displaystyle}
\newcommand{\GAP}{{\tt GAP}}
\renewcommand{\iff}{if and only if}
\newcommand{\kl}{Kazh\-dan--Lusz\-tig}
\newcommand{\klpol}{Kazh\-dan--Lusz\-tig po\-ly\-no\-mial}
\newcommand{\LR}{{\rm LR}}
\renewcommand{\min}{_{\rm min}}
\newcommand{\Rc}{{\cal R}}
\newcommand{\Sc}{{\cal S}}

\begin{document}

\title{Coxeter version 3.0}
\author{Fokko du Cloux}

\maketitle

\begin{flushleft}
Institut Girard Desargues\\
UMR 5028 CNRS\\
Universit\'e Lyon-I\\
69622 Villeurbanne Cedex FRANCE\\
{\tt ducloux@igd.univ-lyon1.fr}
\end{flushleft}

\noindent\coxeter\ is a program for the exploration of combinatorial issues 
related to Coxeter groups and Hecke algebras, with a particular emphasis on
the computation of \klpol s and related questions. It is not a symbolic
algebra system; rather, it is an interface for accessing a direct C++
implementation of the concept of a Coxeter group. Although I have not been
able to fully reach this goal in the current version, the idea is to make the
class (actually, the class hierarchy) of Coxeter groups available in the
form of a C++ library, which could then be used efficiently by other 
programmers.

The program aims for maximum performance, both in terms of speed and in terms
of memory usage; it does not aim for maximal user-friendliness. If your needs
are served by higher-level programs like {\tt GAP/Chevie} or {\tt Maple}, by
all means use those; the aim of \coxeter\ is to pick up where these programs
stop. Particularly {\tt Chevie} includes a nice set of \kl\ routines,
including some which are not implemented in \coxeter. 

Extending the
program is certainly possible (see below), but only for users who are on 
speaking terms with C++ and are willing to walk a little through the {\tt .h}
files. Extension takes place by inserting your own additional code, typically
in the {\tt special.cpp} file, and recompiling (unless you are adding whole
new files, recompiling should just be typing {\tt make}.)

\section{What is wrong with this program}\label{section:wrong}

What is mostly wrong with this program is that is neither C nor C++. In fact,
this program has been my learning ground for C++ (where I used to be a C
programmer). I made the shift with some reluctance, as I tend to love the
simple-minded no-nonsense approach of C, and particularly the blazing
efficiency that you can get out of it. When I finally decided to make the
switch, a sizable amount of C code had already been written; moreover I
couldn't afford to take the time to learn everything I should have about C++
for such a project. Therefore, although largely C++ in spirit, the program
suffers from a number of severe defects and shortcomings from the C++
standpoint~:

\begin{itemize}\itemsep0 pt
\item[$\bullet$]i/o is not C++ at all; it is plain C.
\item[$\bullet$]I didn't make use of the STL, even though the program makes
heavy use of things that are provided by the STL, and which I had reinvented
before even realizing that the STL existed : my class {\tt List} is
essentially STL's {\tt vector}, where I use binary trees I could have used
STL's {\tt set} class, of course I should have used STL's {\tt string}, etc.
\item[$\bullet$]I'm not using exceptions at all; instead I've implemented
my own error handling mechanism, probably not rigorously enough. Badly
handled error conditions have been the main source of program crashes in
my testing.
\end{itemize}

\noindent Memory allocation is another thorny issue. Whether this was due to
clumsy programming on my part, or to the overhead of the default memory 
allocator, when I tried to use the builtin {\tt new}, in presence of heavy
resizing (which happens often in large computations) performance all of a 
sudden became terrible (and I mean terrible~: slowing down by a factor of at 
least ten, maybe more.) Therefore I decided to write my own primitive 
allocator, getting only fairly large (never smaller than $2^{16}$ bytes by 
default) blocks from the system, and never returning them during the lifetime 
of the program. My allocations always are a power of $2$ bytes, and therefore
as much as 50\%\ of the memory might get wasted (although the ratio in
practice is much better), but at least the speed is satisfactory. At that
time I hadn't heard yet about the possibility of specifying a memory
allocator for the STL classes; in any case I will make the switch to STL 
classes only if this issue is resolved.

\section{What the program does}\label{section:does}

One of the main improvements with respect to previous versions of \coxeter\ is
that the program is now able to handle essentially arbitrary Coxeter groups
(provided of course that the computation you require does not overflow your 
system). For convenience, and also because it seems to cover all cases where
significant computations are possible, in this version I require that 
twice the rank of the group not exceed the number of bits in a {\tt long}
on your system (so the rank is limited to $16$ on a $32$-bit machine, and
to $32$ on a $64$-bit machine.) One could argue that $16$ is a bit restrictive,
but certainly $32$ will cover all cases of interest.

\medskip

\noindent The functionalities provided by the program may be classified in a 
number of categories~:

\smallskip

\noindent{\em elementary operations~:} reduced form computations; products;
descent sets; elementary Bruhat order comparison; coatoms.

%\smallskip

%\noindent{\em Bruhat ordering~:} Bruhat intervals; extremal pairs;

\smallskip

\noindent{\em\klpol s~:} individual \klpol s and mu-coef\-fi\-cients for the
ordinary, unequal-parameter and inverse cases; layout of a polynomial
computation in these cases; \kl\ basis elements in the Hecke algebra; 
singular stratification, rational singular locus, ordinary and IH Betti
numbers of a Schubert variety.

\smallskip

\noindent{\em\kl\ cells~:} left, right and two-sided cells for the ordinary
and unequal-parameter cases; ordering on the cells of a finite Coxeter group
in both these cases.

\medskip

\noindent One of the serious issues that have to be faced in computational
mathematics is the sheer size of the output for the computations that 
modern-day computers are able to handle. Dumping the output on a screen is 
definitely not good enough in many cases. I have no real solution for this 
problem; in any case it is clear that more and more output will be handled 
electronically, and will have to be analyzed by other programs before being 
usable by humans.

I offer a number of output formats for output to files : a human-readable one,
a ``terse'' one which is designed to be easily computer-readable, for analysis
or further processing, and \GAP-format. The latter is a first step towards
connecting (some appropriate version of) \coxeter\ with \GAP.

\section{What is missing from this version}\label{section:missing}

A number of things that I would have liked to put in the program are missing.
Foremost are parabolic \klpol s. They are important in their own right, but
also because the parabolic setting comes much closer to capture the real
difficulty of a \kl\ computation. Some computations, say of parabolic \klpol s
in type $E_8$, will not go through in this version even though they are
actually rather small, only because as a preliminary the program tries to
construct an enormous Bruhat interval. If we handled things in a parabolic
setting, the parabolic interval would be constructed instead, which would not
be a problem. This would certainly be my number one project (apart from
improving the code structure) for a future version.

Also, there is nowhere as much stuff on Bruhat intervals as I would like to
have. A {\em lot} more can be said, asked, and done about them. But perhaps
this really would require a different program. It may not be a good idea trying
to do everything in one place.

There is no attempt to handle special groups in any special way. The only way
in which finite groups are special in this program is through the fact that
some computations, such as \kl\ cell computations for instance, are defined
only for them. It is likely that Bruhat order computations could be much 
improved for finite groups; still, my measurements seem to show that even in 
the current state, Bruhat order computations are not all that dominant, so that
at most one could expect to gain a factor of two (and usually much less). In
view of the speed already attained, this doesn't seem worth the effort. Users
who are especially interested in type $A$ should check out Gregory Warrington's
programs (see \url{http://www.math.upenn.edu/~gwar/research/research.html})

\section{How to extend the program}\label{section:extend}

The usual mechanism used by computer algebra programs to extend the
capabilities of the program beyond the predefined commands is to provide
an interpreted programming language that allows the user to write his own
routines.

No such thing is provided here. However, there is a command called {\tt 
special}, defined in {\tt special.cpp}, which executes the {\tt special\_f}
function defined in that file. Currently this function just prints out a
short warning message, but it can be redefined by inserting any code you
wish (and then recompiling the program of course.) It is also easy to
insert new command names executing user-defined functions; it should be
easy to guess what needs to be done by looking at the way the {\tt special}
command is defined in {\tt special.cpp}. Of course this is only useful for
users which are knowledgeable with C++. Also, to do anything useful, you
will want to use the functionalities already provided by the {\tt CoxGroup}
class; this will require reading a few of the {\tt .h} files to see what is
available. In fact, I have tried to make the most useful commands available
already at the level of the {\tt CoxGroup} class (or sometimes in the derived
class {\tt FiniteCoxGroup} if the command doesn't make sense for a general
Coxeter group.) So in principle it would be enough to look there; however,
most of the functions from {\tt CoxGroup} are simply forwarded from the
various components of the class, so it will be necessary to look at a number
of other files to see the definitions and comments for all these functions.

\section{How things are done: word reduction}\label{section:reduction}

The following sections describe in some more detail the algorithms that are
used in the program to perform the computations. We begin with the most
fundamental one, which is word reduction.

What I have done is provide a complete, purely combinatorial implementation
of the minimal root algorithm discovered by Brigitte Brink and Bob Howlett
\cite{brink_howlett:1993}; the usage of minimal roots for word reduction
and normalization, and the practical aspects of the construction of the
set of minimal roots, are nicely discussed in \cite{casselman:2002}.

For any Coxeter system $(W,S)$, Brink and Howlett define a
canonical finite subset $E$ of the set of positive roots of $W$, called the 
set of minimal (or elementary) roots, which contains the simple roots.
Define an action of $S$ on the set $X=E\cup\{+,-\}$, where $+$ and $-$ are
two special symbols, by letting $+$ and $-$ be fixed points, and setting
$s.\a=+$ if $\a$ is a positive root not in $E$, $s.\a=-$ if $\a=\a_s$ is
the simple root corresponding to $s$, and $s.\a=\b\in E$ otherwise. So
we can now view $X$ as a finite state automaton with alphabet $S$ (except
that we do not choose an initial state or a set of accept states at this
point.) Then it is shown in \cite{brink_howlett:1993} that if $s_1\ldots s_p$ 
is a {\em reduced} word in $W$, and if we read this word through the automaton,
starting from the state $\a_s$, the word $s_1\ldots s_ps$ is reduced, unless
we reach the state $-$; moreover, if the generator $s_j$ takes us from state
$\a_t$ to state $-$, we have $s_j=t$ and 
$s_{j+1}\ldots s_ps=ts_{j+1}\ldots s_p$; hence the reduction. It is clear
now how the above automaton can be used to reduce an arbitrary word in
the generators, in at most quadratic time. This algorithm is implemented
in the function {\tt MinTable::prod} in {\tt minroots.cpp}.

Of course it remains to construct the set of minimal roots, together with
the action of the generators above. In \cite{brink_howlett:1993} an
algorithm is described in terms of the standard geometrical realization
of the group. This algorithm has the drawback that it requires finding
the sign of potentially complicated algebraic real numbers. Using the
detailed analysis of minimal roots in \cite{brink:1998}, it is possible
to make the algorithm entirely combinatorial, using only a few explicit
irrationalities (and even then, in a formal fashion.) This is explained
in more detail at the beginning of {\tt minroots.cpp}. The upshot is that
we are able to construct the ``minimal root machine'' for essentially any
Coxeter group. The only possible source of trouble is memory overflow; even
though for the most frequently used Coxeter groups (such as finite or affine
groups) the number of minimal roots is fairly small (typically a few hundred),
it can grow larger for more exotic cases. For ranks $\leq 16$, if the
entries of the Coxeter matrix are not too large, there should not be more
than a few tens of thousands of roots; but for ranks $\leq 32$, millions
of minimal roots should be expected in bad cases. The program will quit if it 
is unable to construct the minimal root table; nothing useful can be done 
without word reduction.

Another problem that is neatly solved with the minimal root machine is the
{\em word problem} for Coxeter groups~: recognizing when two words in the
generators represent the same element. If $s_1\ldots s_p$ is a reduced word
in the generators, and $s$ is a generator such that $s_1\ldots s_ps$ is 
reduced, it turns out that the procedure described above also finds all 
possible ways to insert a generator $t$ in $s_1\ldots s_p$, say after
the $j$-th letter, so that $s_1\ldots s_jts_{j+1}\ldots s_p=s_1\ldots s_ps$.
Define the normal form of an element $w\in W$ to be the lexicographically
smallest reduced expression of $w$ (with respect to a chosen linear ordering
on the set $S$). Then it is known that if $s_1\ldots s_p$ is a normal form,
and $s_1\ldots s_ps$ is reduced (not reduced), the normal form of 
$s_1\ldots s_ps$ is obtained through a suitable insertion (deletion) in 
$s_1\ldots s_p$. Now it is easy to show that the minimal root machine in
fact will find all possible insertion/deletion places in a given word,
for the multiplication by an additional generator $s$; so it is also
possible to construct the normal form of an element using the finite state
automaton described above.

Of course, once we have word reduction it is an easy matter to determine
(though not very efficiently) the descent set of any given element of
the group (recall that the left (right) descent set of $w\in W$ is the set
of $s\in S$ such that $l(sw)<l(w)$ ($l(ws)<l(w)$)), where $l$ is the usual
length function on $W$.

Also, there is an elementary algorithm to decide, given two elements $x$ and
$y$ in $W$, if $x\leq y$ for the Bruhat ordering on $W$. It goes as follows~:
{\it (a)} if $y=e$ (the identity element in $W$), then $x\leq y$ if and only if
$x=e$; {\it (b)} otherwise, choose $s\in S$ such that $l(ys)<l(y)$ (for 
instance, the last element in a given reduced expression of $y$); {\it (c)} 
if $l(xs)<l(x)$, then $x\leq y$ if and only if $xs\leq ys$; otherwise,
$x\leq y$ if and only if $x\leq ys$.

Because of the availability of efficient word reduction, the general
representation of group elements in the program is through {\em reduced}
expressions; this is the {\tt CoxWord} class defined in {\tt coxtypes.h}.
Of course, a word in the generators cannot know that it is reduced; it is
the programmer's responsibility to ensure (using the available reduction
tools if necessary) that words remain reduced at all times. This makes
it possible to implement the {\tt length} function, for instance, simply
by returning the length of the word as a string. On the other hand, I do
not insist that words always be normal forms; this would impose too heavy
an overhead for no great benefit, particularly since we are allowing the
user to redefine the chosen ordering of the generators (this changes all
the normal forms). I treat normal forms as an input/output issue.

An annoying little twist in the program is the following : because in C
the zero-character is used as a string-terminator, it could not be used in
the string-representation of group elements. On the other hand, it is really
a bad idea not to start numbering the generators from zero. So we end up
with the awkward situation that generator \#$s$ is represented by character
$s+1$; this requires some shifting when reading and writing strings. Hence
the distinction between the types {\tt Generator} (numbered from $0$) and
{\tt CoxLetter} (numbered from $1$) in {\tt coxtypes.h}. In C++ strings are
represented with an explicit length, so no special terminator character is
required, and we could use zero directly. Since I'm using C++-style strings
now, I could dispense with {\tt CoxLetter}'s entirely, but haven't had the
courage to take this on yet.

\section{How things are done: Bruhat ordering}\label{section:bruhat}

The elementary operations described above are actually used very little in
the program, because it seems to me that they become prohibitively expensive
as soon as one attempts serious \kl\ computations.

In my setup, every Coxeter group $W$ has an {\em enumerated part} $P$, which is
always required to be a decreasing set (also called a closed set, or an order
ideal) for the Bruhat ordering~: if $y\in W$ belongs to $P$, then all $x\leq y$
also belong to $P$. This just means that we have set up once and for all
a $(1,1)$ correspondence between the integers in $[0,N[$, where $N$ is the 
cardinality of $P$, and the elements of $P$; the only requirement for this 
correspondence is that it be increasing with respect to the Bruhat ordering, 
{\em i.e.} if $x\leq y$ for the Bruhat ordering, then $x$ has a smaller label
than $y$. In particular, the label of the identity element $e$ is necessarily
$0$. Initially the enumerated part is the one-element set $\{e\}$. An
element of the enumerated part of the group may now be represented by a
single number; this is the type {\tt CoxNbr}, defined in {\tt coxtypes.h}.

The following tables are maintained for the enumerated part~: {\it (a)} a
table containing the lengths; {\it (b)} a table containing the left and
right descent sets; actually both are packed into a single long integer,
where the $n$ rightmost bits flag the right descents, and the $n$ next
bits flag the left descents, if $n$ is the rank of the group;
{\it (c)} a table recording the result of right or left multiplication by
a generator (in other words, a table with $N$ rows, each having $2n$ entries);
here a special value {\tt undef\_coxnbr} is used when the multiplication takes
the element outside $P$; {\it (d)} a table which gives for each $x\in P$ the
list of {\em coatoms} of $x$; these are the elements $z<x$ such that
$l(z)=l(x)-1$ --- in principle, this table entirely describes the Bruhat
ordering on $P$. The possibility of packing all descents in one {\tt long} is 
the main reason for the requirement that $2n$ should not exceed the number of 
bits in a {\tt long}.

Note that the data contained in these tables are (more than) sufficient to
define completely the correspondence between $[0,N[$ and the elements of
$P$. Indeed, if an element of $P$ is given, say as a reduced word 
$s_1\ldots s_p$, it is enough to start from $0$, and using the multiplication 
table $p$ times we find the corresponding number. Conversely, if a number
$a\in[0,N[$ is given, we look at the descent set to find a left descent for 
$a$, then apply this to $a$, and continue until we reach $0$. The corresponding
sequence of generators, read left-to-right, will then constitute a reduced
expression of the group element labelled $a$. If we choose the 
smallest left descent (for a given ordering of the generators) at each step, 
we even find the normal form of the element corresponding to $a$.

The {\tt SchubertContext} class, defined in {\tt schubert.h}, maintains the
enumerated part of the group. The main issue is the problem of {\em 
extension}~: given an element $w$ in the group (in the form of a reduced
word, as always), which does not belong to the current enumerated part $P$,
enlarge $P$ so that $P$ contains $w$ (in fact, what we do is replace $P$ by
$P\cup[e,w]$, which is the smallest possible decreasing subset containing both
$P$ and $w$.) An algorithm is described in \cite{du_cloux:2002} which does
exactly that, in a purely ``internal'' fashion~: using only the data contained
in the various tables in $P$, and the knowledge of the Coxeter matrix of $W$,
it is able to find which elements of $[e,w]$ do not currently belong to $P$,
and enlarge $P$ and all its tables accordingly, putting the new elements
on top. This enlargement has the very nice property that the only modifications
that are made to the already existing enumerated part is that some previously
right or left multiplications may become defined; otherwise all existing
references to elements in $P$ remain valid. The function which performs this
extension is the {\tt extendContext} member function of the {\tt CoxGroup}
class.

The same algorithm which is used to build up the enumerated part may be
used to extract a given interval from the identity $[e,w]\subset P$; things
are much simpler here because nothing needs to be constructed, it is only
a matter of lookup. This is done by the member function {\tt extractClosure}
from the {\tt CoxGroup} class; this (or rather the member function from the
{\tt SchubertContext} class that it refers to) is probably the most 
heavily-used function in the program.

\section{How things are done: \klpol s}\label{section:klpols}

I believe that it is clear to anyone who has ever attempted to compute
\klpol s of any substance, that it is essential to remember polynomials
already computed. On the other hand, we do not want too remember too many,
since the size of the table of computed polynomials is usually the
limiting factor for \kl\ computations. It is well-known, already from
the original \kl\ paper \cite{kl:1979}, that the computation of $P_{x,y}$,
for $x\leq y$ in $W$, readily reduces to the case where the two-sided descent 
set $\LR(y)$ is contained in $\LR(x)$; I call such pairs {\em extremal}.

\klpol{} computations are attempted only for elements of $W$ which belong
to the enumerated part $P$ (cf.\ section \ref{section:bruhat}) (in other 
words, the first stage of the computation is extending the enumerated part if 
necessary.) Again a number of tables are maintained. The first one is the
table of extremal pairs~: for each $y$ in $P$, this table contains a 
pointer (initially zero), which when non-zero points to a row
of numbers representing the $x\leq y$ which are extremal w.r.t.\ $y$. The
second one contains for each $y$ in $P$ a pointer to a
row of polynomials (more precisely,
of pointers to polynomials), one for each extremal $x$; finally, for each
$y$ in $P$ a list is maintained which, when initialized,
is guaranteed to contain all $x\leq y$
for which the mu-coefficient $\mu(x,y)$ (the coefficient of highest possible
degree in $P_{x,y}$) is non-zero. In the latter list, ideally we would like
to have an entry exactly when $\mu(x,y)$ is non-zero; however to do this
consistently would require the computation of the full row as soon as the
row is created, which appears at first sight to be rather expensive (but I
think that it would be worthwile to explore this issue further.)

One of the novel features in \coxeter3 is the fact that it will handle
unequal-parameter and inverse \klpol s as well as the ordinary ones. In
each case, the corresponding tables are maintained (although the mu-tables
in the unequal-parameter case are a bit different; one table needs to be
maintained for each generator $s$, and the entries in the table are
(Laurent) polynomials, not numbers.) All these computations reduce to the
same set of extremal pairs, so the extremal-pair tables are shared among
the three contexts, and managed by the {\tt KLSupport} class, defined in
{\tt klsupport.h}. Perhaps it is worth explaining how the extremal lists
are constructed. First we construct the interval $[e,y]$ using 
{\tt extractClosure}, as explained in section \ref{section:bruhat}. The
result is returned in the form of a bitmap. Now the required set of extremal
elements is the subset of $[e,y]$ made up by those elements which are taken
down by each left/right multiplication from the descent set $\LR(y)$. What
we do is for each left or right multiplication by a generator, maintain a
bitmap of the set of elements in $P$ which are taken down (these bitmaps
have to be enlarged at each extension of the enumerated part.) Then to
compute the extremal elements, it is enough to intersect $[e,y]$ with the
appropriate bitmaps, which is a very fast operation, and finally read
the result into a list. One of the nice features of the enumerated-context
setup, by the way, is the completely symmetric treatment of right and
left multiplication; in fact in the program the parameter $s$ takes values
between $0$ and $2n-1$, where $n$ is the rank. If $s<n$ we are dealing
with a right multiplication, otherwise with a left multiplication; but very
rarely is this distinction relevant.

In the \klpol-tables I bear the self-imposed burden of filling the row
for $y$ only if $y$ is smaller than its inverse. Otherwise the lookup
function knows that it has to go over to $y^{-1}$ and get the polynomial
from there, as $P_{x,y}=P_{x^{-1},y^{-1}}$. This does save a sizable amount
of memory, but more importantly the discipline of systematically going over
to $y^{-1}$ when possible seems to drive the recursion down faster than it
would otherwise go, for some reason which I don't fully understand yet. 
However, this setup introduces 
complications which mar the elegance of the program considerably, and I'm
always on the verge of renouncing it. Since the mu-tables are in fact
rather small (more precisely, the number of non-zero terms in them is small,
so that they could in theory be made small), I have not imposed this
additional constraint on them.

The approach taken by \coxeter3 to the computation of \klpol s is rather
different from that of its predecessor \coxeter1, in that it computes
the polynomials on demand, whereas \coxeter1 would compute the full table
of \klpol s for the group before doing anything else. (This has the advantage
that the computation can be organized with optimal efficiency; for instance
only non-zero mu's need enter the picture, and a lot of searching can be
avoided. So for these full-table computations, when they are possible, the old
program will still be faster.) In the case of \coxeter3, only the row for
the identity is filled at startup. Then if a polynomial $P_{x,y}$ is required,
and we are not in a trivial case where the answer can be given without
computation, first the extremal list for $y$ is constructed (if it was not
already available), and the row of polynomials for $y$ is allocated and
initialized with zero-pointers; then the recursion formula for $P_{x,y}$ is
mapped out, the corresponding polynomials and mu-coefficients are computed
if not already available, the result is found, and its address is looked up
in the table of existing \klpol s, adding the new polynomial if necessary;
then a reference to that table entry is returned. In the course of the
computation, positivity of the coefficients is checked; a negative coefficient
will cause an immediate exit with a (congratulatory) error message. The
recursion I use is the original recursion formula from \cite{kl:1979}; to my 
knowledge this is still the most efficient one, and will automatically take 
advantage of all the simplifications that I'm aware of in special cases.

If a whole row of polynomials is requested (this will be the case, for 
instance, when an element of the \kl\ basis of the Hecke algebra of $W$ is 
desired, or when one studies the singularities of the Schubert variety
corresponding to $y$), then the computation can be done quite a bit more
efficiently. In this case, all computations are done by full rows (even though
this might lead to the computation of some more polynomials than strictly
necessary), and also it turns out that all calls to individual Bruhat order
comparisons, which take up a good deal of the computing time for individual
polynomial computations, can be avoided. Even though this is elementary, it is 
perhaps worth explaining as I only realized it very recently. Recall that the 
recursion formula for $P_{x,y}$, for an $s\in S$ such that $ys<y$ and $xs<x$, 
amounts to adding $P_{xs,ys}$ and $qP_{x,ys}$, and then subtracting
$\mu(z,ys)q^{\frac{1}{2}(l(y)-l(z))}P_{x,z}$ for all $x\leq z<ys$ such that 
$zs<z$. The first two terms can be gotten from the row for $ys$ (and clearly, 
as $x$ varies, most if not all of that row will be used, so that one might as 
well compute it all.) The set of $z<ys$ s.t. $zs<z$ can be gotten from one call
to {\tt extractClosure}, and intersection with the ``downset'' for right 
multiplication by $s$ as explained above for the construction of extremal 
lists. Then we check $\mu(z,ys)$; this requires one row in the mu-table, which 
we get for free since it is deduced from the row in the \kl\ table which we 
already have. If $\mu(z,ys)$ is zero, which will happen most of the time, we 
move on to the next $z$. The annoying condition is $x\leq z$. But this is 
avoided as follows~: since we are computing a full row, we are considering {\em
all} such $x$'es; so we extract $[e,z]$ again with {\tt extractClosure}, 
intersect with the extremal list for $y$, and get the set of $x$'es for which 
we have to do the subtraction of $\mu(z,ys)q^{\frac{1}{2}(l(y)-l(z))}P_{x,z}$. 
Thanks to this trick which is much better than what I was doing in \coxeter1, 
some of the speed-differential between the two programs is removed.

\section{How things are done: $\mu$-coefficients}\label{section:mucoeffs}

As explained in the previous section, when one is computing whole rows of
polynomials, or the full table of \klpol s, the $mu$-coefficients come for
free~: they can be read off from the corresponding polynomials, which are
available when the $\mu$'s are needed. The situation is rather different
when one aims for the computation of a single polynomial, and one wishes,
as I try to do in this program, to compute only the strictly necessary
ingredients. It would then be rather wasteful to compute a whole polynomial
(and of course many others because of all the recursions that might be
triggered) when only the $\mu$-part is required. This is even more so when
we are only interested in the $\mu$-table, for instance when we wish to
determine the $W$-graph of the group, or its decomposition into \kl\ cells.

It turns out that for the ordinary \klpol s, the computation of the $\mu$'s
affords some remarkable simplifications which makes it several orders of 
magnitude easier than the computation of the corresponding polynomials. I
should confess, by the way, that I became fully aware of this fact only
rather late in the construction of the program, and that I'm not quite sure
I assessed it yet to the full. Just as in \cite{kl:1979}, Corollary 4.3,
one sees that the for any given $x\leq y$ in $W$, the ordinary \klpol\ 
$P_{x,y}$ and the inverse one $Q_{x,y}$ share the same $\mu$-coefficient.
So the same simplifications hold for the inverse $\mu$-coefficients as well.
(In fact, the two computations could actually share the same $\mu$-table,
although this is not currently the case in the program.)

Here is how it goes. Let $x\leq y\in W$. When the length difference between
$x$ and $y$ is even, we already know that $\mu(x,y)=0$; when the length 
difference is one ({\em i.e.}, $x$ is a coatom of $y$), $\mu(x,y)=1$. So
assume that $l(y)-l(x)$ is odd and $>1$. If $x$ is not extremal w.r.t. $y$,
we also know that $\mu(x,y)=0$; so we may also assume that $x$ is ectremal
w.r.t. $y$. These easy reductions, by the way, determine the default
allocation of a row in the $\mu$-table~: when no further reductions are 
available, we allocate one entry for each $x$ satisfying the above conditions.
Now look at the recursion formula for $\mu(x,y)$ obtained by taking the term
of degree $\ds{\frac{1}{2}(l(y)-l(x)-1)}$ in the recursion formula for 
$P_{x,y}$. We start with $\mu(xs,ys)$, then add the coefficient in degree
$\ds{\frac{1}{2}(l(ys)-l(x)-2)}$ in $P_{x,ys}$ (note that the length difference
between $x$ and $ys$ is even, so this is the highest-possible degree term in
$P_{x,ys}$, but not a $\mu$-coefficient), and subtract the sum of all
$\mu(x,z)\mu(z,ys)$, where $x<z<ys$ is such that $zs<z$ (and of course we
may assume that $l(z)-l(x)$ is odd.) Notice already that the only term which
prevents this formula from being a recursion internal to the $\mu$-table is
the one coming from $P_{x,ys}$. But if it is the case that there is a generator
$t$ such that $yst<ys$, $xt>x$ (following our usual conventions we write
products on the right, but when $t>n$ this is in fact multiplication on the
left), in other words, if $x$ is not also extremal w.r.t. $ys$, then we see
that the term we need from $P_{x,ys}$ is in fact $\mu(xt,ys)$, and
the recursion takes place entirely within the $\mu$-table. But much more is
true~: following \cite{kl:1979}, section 4, one sees that in the subtracted
sum there are at most two non-vanishing terms. Indeed, if $l(ys)-l(z)=1$,
then we will have $zt<z$ unless $z=yst$; and if $zt<z$ we have $\mu(x,z)=0$
because $x$ is then not extremal w.r.t.\ $z$. If $l(z)-l(x)=1$, we see 
symmetrically that $zt>z$ unless $z=xt$; and if $zt>z$, $z$ is not extremal
w.r.t.\ $ys$. And if both length differences are $>1$, we have $\mu(x,z)=0$
if $zt<z$, and $\mu(z,ys)=0$ if $zt>z$. Since $\mu$-coefficients are $1$ when
the length difference is one, we see that the subtracted sum reduces at most 
to the two terms $\mu(x,yst)+\mu(xt,ys)$. The additional condition $zs<s$ 
yields the final result :
$$
\mu(x,y)=\cases{\mu(xs,ys)&if $xts<xt$, $ysts>yst$\cr
\mu(xs,ys)-\mu(x,yst)&if $xts<xt$, $ysts<yst$\cr
\mu(xs,ys)+\mu(xt,ys)&if $xts>xt$, $ysts>yst$\cr
\mu(xs,ys)+\mu(xt,ys)-\mu(x,yst)&if $xts>xt$, $ysts<yst$}
$$
Since $x$ is extremal w.r.t.\ $y$, the condition $xt>x$ implies $yt>y$; but
it is easy to see that $yst<ys$ is then only possible if $s$ and $t$ do {\em
not} commute; in particular they correspond to multiplications which take
place on the same side. If the coefficient $m(s,t)$ of the Coxeter matrix of
$W$ is equal to three, there is a further simplification. It is then easy
to see that the case $ysts<yst$ cannot occur. Moreover if $xts>xt$ we
have $xs<x<xt<xts$ so that $xs$ is the shortest element in the coset of $x$ for
the parabolic subgroup of $W$ generated by $s$ and $t$, and $xst>xs$, hence
$xs$ is not extremal w.r.t.\ $ys$, and $\mu(xs,ys)=0$. So the formul\ae\
simplify to~:
$$
\mu(x,y)=\cases{\mu(xs,ys)&if $xts<xt$\cr\mu(xt,ys)&if $xts>xt$\cr}
$$
The upshot is that the full recursion formula involving the extraction of
the interval $[x,ys]$ has to be called only if $\LR(x)$ contains not only
$\LR(y)$, but $\LR(ys)$ as well for {\em every} $s\in\LR(y)$. This will
happen only in a very small number of cases. On the other hand, when it does
happen, the computation of $\mu(x,y)$ will trigger a great many recursive
calls, so that in the end the gain is not as big as one might expect at first.

Another aspect of things is that one might use these remarks to condense
the $\mu$-table considerably, storing only the ``double-extremal'' pairs,
particularly in the case of simply-laced groups.
This would certainly be the way to go if one were to attempt the computation
of, say, the full $W$-graph of a group like ${\rm E}_7$. I plan to explore
these issues further using a suitably modified version of the program.

\section{How things are done: \kl\ cells}\label{section:klcells}

Another feature missing from \coxeter1 which is provided by \coxeter3 is the
computation of (left, right and two-sided) \kl\ cells. This is done only for
finite groups, as I don't see as yet a rigorous way of doing it for infinite
groups even when, as in the case of affine groups, it is known that the set
of cells is finite.

One brute-force way of computing cells is to compute the full $\mu$-table of
the group, get from there the full $W$-graph, and then notice that the problem
is an instance of a classic computer algebra problem, {\em viz.} computing
equivalence classes in an oriented graph. (I wish to thank Bill Casselman for
pointing this out, and for explaining to me the Tarjan algorithm which
performs this computation.) In fact, if in addition to the partition of the
group in cells, one wishes to recover the poset structure on the set of cells
(which for finite Weyl groups is isomorphic to the poset of primitive ideals
in the enveloping algebra of the corresponding semisimple Lie algebra, with
trivial infinitesimal character), then it is not very likely that one can get
away with much less than that. But if only the partition is required, much
less needs to be done.

Here is how one might go about it. We will say that two elements $x$ and $y$
in $W$ are in the same right descent class, if their right descent sets are
equal. It is known already from \cite{kl:1979} that all elements of a given 
left cell are in the same right descent class.
Moreover, there is a set of partially defined operations on the group, the 
so-called $*$-operations, which preserve left cells. One operation is defined
for each pair of non-commuting elements $s,t$ in the group. The idea is to
look at the right cosets for the parabolic subgroup generated by $s$ and $t$.
If we set $m=m(s,t)$, then each coset has $2m$ elements, one of minimal
length, one of maximal length, and two in each intermediate lengths. The
domain of the operation $*_{s,t}$ is the set of elements in $W$ which are
neither of minimal nor of maximal length in their coset (equivalently, this
means that $R(w)\cap\{s,t\}$ has exactly one element.) For each coset $C$
in $W$ with minimal length representative $x\min$ we define the two 
$\{s,t\}$-chains of $C$ to be the $(m-1)$-element sets 
$C_s=\{x\min s,x\min st,\ldots\}$ and $C_t=\{x\min t,x\min ts,\ldots\}$.
Then each element $w$ in the domain of $*_{s,t}$ is contained in exactly one
$\{s,t\}$-chain $\{x_1,\ldots,x_{m-1}\}$. If $j\in\{1,\ldots,m-1\}$ is the 
index such that $w=x_j$, we set $w*_{s,t}=x_{m-j}$. In particular, $*_{s,t}$
permutes each $\{s,t\}$-chain. Clearly the domain of each $*_{s,t}$ is a
union of right descent classes, and therefore of left cells. Then it is known
\cite{lusztig:1985} that $*_{s,t}$ takes each left cell in its domain to 
another left cell.

This allows us to refine the partition of $W$ in right descent classes as
follows. We define a sequence of partitions $(\Rc_k)_{k\geq 1}$ of $W$ by
letting $\Rc_1$ be the partition in right descent classes, and saying that
$x$ and $y$ have same class in $\Rc_k$, $k>1$, \iff\ they belong to the same 
class in $\Rc_{k-1}$, and for each pair $\{s,t\}$ of non-commuting generators 
in the group such that $*_{s,t}$ is defined for $x$ (and hence for $y$),
$x*_{s,t}$ and $y*_{s,t}$ also belong to the same class in $\Rc_{k-1}$. We
then define $\Rc_\infty$ by saying that $x$ and $y$ belong to the same class
in $\Rc_\infty$ \iff\ they belong to the same class in $\Rc_k$ for all 
$k\geq1$. Clearly the partition of $W$ in left cells refines the partition 
$\Rc_\infty$; in the case of a finite Weyl group, the partition $\Rc_\infty$ is
the generalized $\tau$-partition introduced by David Vogan \cite{vogan:1979}.

It turns out that it is not hard to compute the partition $\Rc_\infty$, by
an algorithm rather similar to the one which will construct the minimal
automaton corresponding to any given finite state automaton (see for
instance \cite{aho_seti_ullman:1986} algorithm 3.6.) On the other hand,
there is also an easy lower approximation to the partition of $W$ in left
cells. Indeed, it is easy to see that each left cell $C$ contains a whole left
$\{s,t\}$-string as soon as it contains one of its elements. Denote $\Sc$ the 
smallest equivalence relation on $W$ which is compatible with $\{s,t\}$-strings
for all non-commutaing pairs $\{s,t\}$. Again it is not hard to compute the
partition $\Sc$.

Our approach to left cells is to first compute the partitions $\Sc$ and 
$\Rc_\infty$; the former is a refinement of the latter. Any class for 
$\Rc_\infty$ which is also a class for $\Sc$ must be a left cell; for the
remaining $\Rc_\infty$ classes we compute the $W$-graph of the class and
get the left cells from there. It is known (\cite{kl:1979} section 5) that
for type ${\rm A}$ the partitions $\Rc_\infty$ and $\Sc$ coincide; this seems 
also to be always the case in type ${\rm B}$. So in these cases the cell
partition can be determined without any \kl\ computations whatsoever!

Right cells can of course be deduced from left cells by inversion. I have to
confess that I haven't found the time to study up on two-sided cells as much
as I would have liked; currently their determination is implemented really
brute force-like from the full two-sided $W$-graph of the group. This is
certainly a point which should be improved in the future. Another thing that
is currently missing from the program is the computation of Lusztig's 
$a$-function. Hmm...

We now describe the member functions of the {\tt FiniteCoxGroup} class which
are available for accessing the various partitions defined in this section.
Only the actual \kl\ cells have a corresponding command in the default
command interface, and can be printed out in various formats. The standard
way of representing a partition of a set in $p$ classes, is through a function
taking values in the set $\{0,\ldots,p-1\}$; and a general function on a set 
with $N$ elements is just a sequence of $N$ numbers. The various partition
functions all return references to objects of type {\tt Partition}, defined
in {\tt bits.h}; in addition to the actual partition function, this class
contains the number of classes of the partition, and a number of member
functions for sorting and traversing partitioned sets conveniently. Since
some of these partitions may be rather expensive to compute, we avoid
computing them more than once; the finite group structure contains predefined
partition objects, initially empty, which are filled when first required. So
after the first call, access should be instantaneous.

\begin{itemize}\itemsep0pt
\item[$\bullet$]{\tt lCell, rCell, lrCell}~: the partitions in ordinary \kl\ 
cells (these partitions may be printed out using the commands {\tt lcells}, 
{\tt rcells}, {\tt lrcells} from the default command interface);
\item[$\bullet$]{\tt lUneqCell, rUneqCell, lrUneqCell}~: the partitions in 
\kl\ cells for unequal parameters (these partitions may be printed out
using the commands {\tt lcells}, {\tt rcells}, {\tt lrcells} as above, after 
passing to ``unequal parameter mode'' using the command {\tt uneq});
\item[$\bullet$]{\tt lDescentPartition, rDescentPartition}~: the partitions
in left and right descent classes;
\item[$\bullet$]{\tt lGeneralizedTau, rGeneralizedTau}~: the partitions in
equivalence classes for the generalized tau-invariant (the relation 
$\Rc_\infty$);
\item[$\bullet$]{\tt lStringPartition, rStringPartition}~: the partitions
for the equivalence relation $\Sc$ defined above.
\end{itemize}

\begin{thebibliography}{1}

\bibitem{aho_seti_ullman:1986}
A.V. {Aho, R. Seti and J.D. Ullman}.
\newblock {\em {Compilers}}.
\newblock Addison-Wesley, Reading, Massachussets, 1986.

\bibitem{brink:1998}
B.~Brink.
\newblock {The set of dominance-minimal roots}.
\newblock {\em {J. Algebra}}, {\bf 206}:371--412, 1998.

\bibitem{brink_howlett:1993}
B.~Brink and D.~Howlett.
\newblock {A finiteness property and an automatic structure for Coxeter
  groups}.
\newblock {\em {Math. Ann.}}, {\bf 296}:179--190, 1993.

\bibitem{casselman:2002}
W.~Casselman.
\newblock {Computation in Coxeter groups. I. Multiplication.}
\newblock {\em {Electron. J. Combin.}}, {\bf 9}(1):Research Paper 25, 22 pages,
  2002.

\bibitem{du_cloux:2002}
{F. du} Cloux.
\newblock {Computing Kazhdan-Lusztig polynomials in arbitrary Coxeter groups}.
\newblock {\em {Experiment. Math.}}, {\bf 11}(3):387--397, 2002.

\bibitem{kl:1979}
D.~Kazhdan and G.~Lusztig.
\newblock {Representations of Coxeter groups and Hecke algebras}.
\newblock {\em {Invent. Math.}}, {\bf 53}:165--184, 1979.

\bibitem{lusztig:1985}
G.~Lusztig.
\newblock {Cells in Affine Weyl Groups}.
\newblock In {\em {Algebraic Groups and relaed topics (Kyoto/Nagoya 1983)}},
  volume~6 of {\em Adv. Stud. Pure Math.}, pages 255--287, Amsterdam, 1985.
  North-Holland.

\bibitem{vogan:1979}
D.A. {Vogan Jr.}
\newblock {A generalized $\tau$-invariant for the primitive spectrum of a
  semisimple Lie algebra}.
\newblock {\em {Math. Ann.}}, {\bf 242}:209--224, 1979.

\end{thebibliography}

\end{document}
